\documentclass{article}
\usepackage[utf8]{inputenc}
\usepackage{enumitem}

\begin{document}

\section*{Multiple Choice Questions on Python Variables, Data Types, and Operators}

\begin{enumerate}[leftmargin=0pt,label=\arabic*.]

  \item Which of the following is a valid Python variable name?
  \begin{enumerate}[label=(\alph*)]
    \item \texttt{1st\_value}
    \item \texttt{first-value}
    \item \texttt{\_first\_value}
    \item \texttt{first value}
  \end{enumerate}

  \item What is the output (and its type) of the expression \texttt{2 + 3 * 4}?
  \begin{enumerate}[label=(\alph*)]
    \item \texttt{20} (int)
    \item \texttt{14} (int)
    \item \texttt{5.0} (float)
    \item \texttt{14.0} (float)
  \end{enumerate}

  \item What is the data type of the result of \texttt{'3' + '4'}?
  \begin{enumerate}[label=(\alph*)]
    \item \texttt{int}
    \item \texttt{str}
    \item \texttt{float}
    \item \texttt{list}
  \end{enumerate}

  \item What does the Boolean expression \texttt{not (True and False)} evaluate to?
  \begin{enumerate}[label=(\alph*)]
    \item \texttt{True}
    \item \texttt{False}
    \item \texttt{None}
    \item Raises a \texttt{SyntaxError}
  \end{enumerate}

  \item What are the results of the expressions \texttt{7 // 3} and \texttt{7 \% 3}, respectively?
  \begin{enumerate}[label=(\alph*)]
    \item \texttt{2} and \texttt{1}
    \item \texttt{2.33} and \texttt{1.0}
    \item \texttt{2} and \texttt{0}
    \item \texttt{3} and \texttt{1}
  \end{enumerate}

  \newpage

  \item What is the result of the operation \texttt{5 < 1}?
  \begin{enumerate}[label=(\alph*)]
    \item \texttt{True}
    \item \texttt{False}
  \end{enumerate}


  \item Given \texttt{x = 10}, what will be the value of \texttt{x} after evaluating \texttt{x *= 3}?
  \begin{enumerate}[label=(\alph*)]
    \item \texttt{3}
    \item \texttt{10}
    \item \texttt{30}
    \item \texttt{100}
  \end{enumerate}

  \item Which of the following data types in Python is mutable?
  \begin{enumerate}[label=(\alph*)]
    \item \texttt{tuple}
    \item \texttt{str}
    \item \texttt{list}
    \item \texttt{int}
  \end{enumerate}

\end{enumerate}

\end{document}

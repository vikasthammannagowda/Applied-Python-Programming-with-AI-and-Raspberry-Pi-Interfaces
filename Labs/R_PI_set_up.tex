\documentclass[11pt]{article}
\usepackage[margin=1in]{geometry}
\usepackage{titlesec}
\usepackage{hyperref}
\usepackage{enumitem}
\usepackage{booktabs}
\usepackage{longtable}
\usepackage{fancyhdr}
\usepackage{setspace}
\usepackage{xcolor}
\usepackage{microtype}
\usepackage{mathpazo}
\usepackage{caption}
\usepackage{tcolorbox}

\usepackage{graphicx} % Required for including images
\usepackage{caption} % Optional: for better captions
\hypersetup{
  colorlinks=true,
  linkcolor=blue,
  urlcolor=teal,
  pdftitle={Data Visualization and Storytelling with AI Support},
  pdfauthor={Instructor}
}


\tcbuselibrary{listings,skins}  

\usepackage{changepage}

\usepackage{caption}
\usepackage{draftwatermark}
\usepackage{environ}

\usepackage{lmodern} % Optional: Use a scalable font family
\usepackage{scalefnt} % Optional: Allow smooth font scaling

\usepackage{siunitx} % ohm symbol

\usepackage{fancyvrb}     % for the Verbatim environment
\hypersetup{
  colorlinks=true,
  linkcolor=blue,
  urlcolor=teal,
  pdftitle={Data Visualization and Storytelling with AI Support},
  pdfauthor={Instructor}
}

% Customize watermark
\SetWatermarkText{\scalefont{1.5}VIKAS} % Use scalefont for scaling
\SetWatermarkScale{.2} % Adjust scale to avoid very large fonts
\SetWatermarkColor[gray]{0.7} % Light gray (80% white)
\SetWatermarkAngle{30}
%\SetWatermarkHorCenter{8cm} % Move right by 3 cm
%\SetWatermarkVerCenter{18cm} % Move up by 12 cm
%%%%%%%%%%%%%%%%%%%%%%%%%%%%%%%%%%%%%%%%%%%%%%%%%%

% Header/footer
\pagestyle{fancy}
\fancyhf{}
\lhead{Data Visualization \& Storytelling with AI}
\rhead{Syllabus}
\cfoot{\thepage}

% Section formatting
\titleformat{\section}{\large\bfseries}{\thesection}{1em}{}
\titleformat{\subsection}{\normalsize\bfseries}{\thesubsection}{1em}{}

% Tight lists
\setlist{nosep}

\begin{document}

\begin{center}
  {\LARGE \textbf{Case Study 3: Indoor Climate Monitoring and Visualization with Raspberry Pi}}\\[0.5em]
  {\Large \textbf{DAT-230 Data Visualization \& Storytelling with AI}}\\[0.5em]
  %{\normalsize R Programming Track — No Programming Background Required}\\[1em]
  {\small Instructor: Dr. Vikas Thammanna Gowda \quad\quad Semester: Fall 2025}\\
  {\small Contact: \texttt{vthammannagowda@champlain.edu} \quad\quad Office Location: West Hall 100}\\
    {\small Office Hours: TBD}\\
\end{center}


\noindent \textbf{Case Study 3: Weather and Climate Patterns}\\
\noindent \textbf{Assigned Date:} (Insert date here) \hfill \textbf{Due Date:} (Insert date here)\\[2ex]


\section*{LLM Prompts for Raspberry Pi Setup}


\subsection*{Prompt 1: Dependency Installation}
  
 \texttt{You are a Raspberry Pi expert. \\
Please guide me step-by-step to update the package index and install 
Python3, pip, venv, and the sensor libraries (Adafruit\_DHT, 
adafruit-circuitpython-tsl2561, \\
adafruit-circuitpython-ads1$\times$15) 
on Raspberry Pi OS. 
Explain each 
command's 
purpose 
and how to verify successful installation.}
  

\subsection*{Prompt 2: Enable I\textsuperscript{2}C}

 \texttt{You are a Raspberry Pi configuration assistant. \\
Explain how to enable I2C on Raspberry Pi OS using raspi-config, including menu 
navigation and verification of I2C status.
 }

\subsection*{Prompt 3: MQ135 Instantiation}
\texttt{
You are an Adafruit sensor library tutor. Show me how to correctly instantiate an AnalogIn channel for the MQ135 sensor using adafruit-ads1x15 in Python, including import statements and usage.
 }

\subsection*{Prompt 4: LDR Integration}
\texttt{
You are a hardware integration instructor. 
Describe how to wire an LDR in a resistor divider to an ADS1115 channel, 
and update the read\_light\_lux() function to read and return raw or normalized LDR values in Python.
 }

\subsection*{Prompt 5: MQ135 Calibration}
\texttt{
You are a data calibration guide. 
Explain how to record a clean-air baseline for MQ135 at startup, 
implement normalization (normalized = (raw - baseline)/(max\_expected - baseline)), 
and log both raw and normalized readings.
}

\subsection*{Prompt 6: Time Synchronization}
\texttt{
You are a system administrator. Instruct me on installing and configuring NTP or systemd-timesyncd on Raspberry Pi and how to verify the clock synchronization status.
}

\subsection*{Prompt 7: Deployment Metadata}
\texttt{
You are a deployment documentation assistant. Show me how to structure a JSON metadata section to record placement details, intentional bias notes (e.g., “sunlight exposure 14:00–16:00”), and calibration offsets before a data run.
}


\section*{Immediate Next Steps on the Raspberry Pi}

\begin{itemize}
  \item \textbf{Install dependencies} (for Debian‐based Raspberry Pi OS):  
    \begin{verbatim}
sudo apt update
sudo apt install python3-pip python3-venv
pip install Adafruit_DHT adafruit-circuitpython-tsl2561 adafruit-circuitpython-ads1x15
    \end{verbatim}

  \item \textbf{Enable I\textsuperscript{2}C:}  
    In \texttt{sudo raspi-config} under “Interface Options,” turn on I\textsuperscript{2}C if using ADS1115 or TSL2561.

  \item \textbf{Fix the MQ135 channel instantiation:}  
    Replace the placeholder in \texttt{read\_mq135\_raw()} with a real AnalogIn. For example:
    \begin{verbatim}
from adafruit_ads1x15.analog_in import AnalogIn
chan = AnalogIn(ads, ADS.P0)   % or AnalogIn(ads, 0) as needed
return chan.value
    \end{verbatim}

  \item \textbf{Swap in an LDR (optional):}  
    \begin{itemize}
      \item Wire the LDR in a resistor divider to an ADS1115 channel.  
      \item Update \texttt{read\_light\_lux()} to read from that channel and log raw or normalized values (note: it won’t be true lux without calibration).
    \end{itemize}

  \item \textbf{MQ135 calibration / mapping:}
    \begin{itemize}
      \item Read a clean‐air baseline at startup.  
      \item Optionally normalize:  
        \begin{verbatim}
normalized = (raw - baseline) / (max_expected - baseline)
        \end{verbatim}
      \item Always also record the raw value for later drift correction.
    \end{itemize}

  \item \textbf{Ensure time synchronization:}
    \begin{itemize}
      \item Install and enable \texttt{ntp} or \texttt{systemd-timesyncd}.  
      \item The script should timestamp readings immediately upon sensor acquisition.  
      \item For multi‐Pi deployments, point all devices at the same NTP server.
    \end{itemize}

  \item \textbf{Deployment descriptor / bias annotation:}  
    Before starting a data run, edit your JSON (or script) metadata to record:
    \begin{itemize}
      \item Placement details (height, distance to window/vent).  
      \item Any intentional bias (e.g., “sensor exposed to direct sunlight 14:00–16:00 to illustrate radiative heating”).  
      \item Calibration offsets or notes.
    \end{itemize}
\end{itemize}





\end{document}
